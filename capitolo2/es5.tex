\subsection{Esercizio 5}
Scrivere function Matlab distinte che implementino efficientemente i seguenti metodi
per la ricerca degli zeri di una funzione $f(x)$:
\begin{itemize}
    \item il metodo di Newton;
    \item il metodo delle secanti;
    \item il metodo di Steffensen:
\end{itemize}
\begin{eqnarray*}
    x_{n+1}=x_n - \frac{f(x_n)^2}{f(x_n + f(x_n)) - f(x_n)}, & & \mbox{n=0,1,...}
\end{eqnarray*}
Per tutti i metodi, utilizzare come criterio di arresto
\[
    \abs{x_{n+1} - x_n} \leq tol * (1 + \abs{x_n})
\]
essendo $tol$ una opportuna tolleranza specificata in ingresso. Curare particolarmente la robustezza
del codice.
\newline \textbf{Soluzione:}

\begin{itemize}
    \item Metodo di Newton
          \lstinputlisting{matlab/capitolo2/newton.m}
    \item Metodo delle secanti
          \lstinputlisting{matlab/capitolo2/secanti.m}
    \item Metodo di Steffensen
          \lstinputlisting{matlab/capitolo2/steffensen.m}
\end{itemize}
