\subsection{Esercizio 7}
Utilizzare le $function$ del precedente esercizio per determinare una approssimazione
della radice della funzione
\[
    f(x) = \left[x - cos(\frac{\pi}{2}x)\right]^3,
\]
per $tol = 10^{-3}, 10^{-6}, 10^{-9}, 10^{-12},$ partendo da $x_0 = 1$
(e $x_1 = 0.99$ per il metodo delle secanti). Tabulare i risultati,
in modo da confrontare le iterazioni richieste da ciascun metodo. Commentare
i risultati ottenuti.
\newline \textbf{Soluzione:}












Le nuove funzioni utilizzate in  questo esercizio sono:
\begin{itemize}
    \item Metodo di Newton modificato
          \lstinputlisting[language=Matlab]{capitolo2/newtonmod.m}
    \item Metodo delle accelerazioni di Aitken
          \lstinputlisting[language=Matlab]{capitolo2/aitken.m}
\end{itemize}
La radice nulla della funzione $f(x)=x^2tan(x)$ ha molteplicità m = 3, in quanto $0$ annulla due volte il termine
$x^2$ e una volta il termine $tan(x)$.


\begin{table}[h]
    \renewcommand\arraystretch{2}
    \begin{tabular}{|l l l l|}
        \hline
        Tolleranza & Newton                          & Newton modificato               & Aitken                          \\
        \hline
        $10^{-3}$  & 1.99400296195610$\cdot10^{-3}$  & 1.32348898008484$\cdot10^{-23}$ & 3.72603946110722$\cdot10^{-24}$ \\
        $10^{-6}$  & 1.34922220938115$\cdot10^{-6}$  & 1.32348898008484$\cdot10^{-23}$ & 3.72603946110722$\cdot10^{-24}$ \\
        $10^{-9}$  & 1.36940553054800$\cdot10^{-9}$  & 0                               & 2.93579661656743$\cdot10^{-39}$ \\
        $10^{-12}$ & 1.38989077859525$\cdot10^{-12}$ & 0                               & 2.93579661656743$\cdot10^{-39}$ \\
        \hline
    \end{tabular}
    \caption{valori approssimati da newton, newton modificato e aitken(dati raccolti in \nameref{cod:7})}
    \label{tab:7}
\end{table}
\newpage
\begin{figure}[h]
    \includegraphics[scale=0.7]{capitolo2/iter2.png}
    \caption{iterazioni richieste}
    \label{fig:es7}
\end{figure}
Il metodo di newton classico perde la convergenza quadratica, essendo la radice cercata di molteplicità multipla. Il metodo di newton modificato e il metodo di aitken convergono
molto più rapidamente e newton modificato riesce anche a trovare la radice esatta.