\subsection{Esercizio 24}
Costruire una function Matlab che, dato in input $n$, restituisca i pesi della quadratura
della formula di Newton-Cotes di grado $n$. Tabulare, quindi, i pesi delle formule di grado
1, 2, \dots, 7 e 9 (come numeri razionali).
\newline \textbf{Soluzione:}

\lstinputlisting{capitolo5/es24_NewtonCotesWeights.m}
Eseguendo lo script \nameref{cod:24} si ottengono i risultati contenuti nella tabella \ref{tab:24}.
\begin{table}[h]
    \centering
    \renewcommand\arraystretch{2}
    \begin{tabular}{| c | c |}
        \hline
        n & pesi                                                                                                                                                                                      \\
        \hline
        1 & $\frac{1}{2}$ $\frac{1}{2}$                                                                                                                                                               \\
        2 & $\frac{1}{3}$ $\frac{4}{3}$ $\frac{1}{3}$                                                                                                                                                 \\
        3 & $\frac{3}{8}$ $\frac{9}{8}$ $\frac{9}{8}$ $\frac{3}{8}$                                                                                                                                   \\
        4 & $\frac{14}{45}$ $\frac{64}{45}$ $\frac{8}{15}$ $\frac{64}{45}$ $\frac{14}{45}$                                                                                                            \\
        5 & $\frac{95}{288}$ $\frac{125}{96}$ $\frac{125}{144}$ $\frac{125}{144}$ $\frac{125}{96}$ $\frac{95}{288}$                                                                                   \\
        6 & $\frac{41}{140}$ $\frac{54}{35}$ $\frac{27}{140}$ $\frac{68}{35}$ $\frac{27}{140}$ $\frac{54}{35}$ $\frac{41}{140}$                                                                       \\
        7 & $\frac{108}{355}$ $\frac{810}{559}$ $\frac{343}{640}$ $\frac{649}{536}$ $\frac{649}{536}$ $\frac{343}{640}$ $\frac{810}{559}$ $\frac{108}{355}$                                           \\
        9 & $\frac{130}{453}$ $\frac{1374}{869}$ $\frac{243}{2240}$ $\frac{307}{158}$ $\frac{704}{1213}$ $\frac{704}{1213}$ $\frac{307}{158}$ $\frac{243}{2240}$ $\frac{1374}{869}$ $\frac{130}{453}$ \\
        \hline
    \end{tabular}
    \caption{pesi delle formule Newton-Cotes}
    \label{tab:24}
\end{table}
