\subsection{Esercizio 24}
Costruire una function Matlab che, dato in input $n$, restituisca i pesi della quadratura
della formula di Newton-Cotes di grado $n$. Tabulare, quindi, i pesi delle formule di grado
1, 2, \dots, 7 e 9 (come numeri razionali).
\newline \textbf{Soluzione:} \newline
% \lstinputlisting[language=Matlab]{capitolo5/trapecomp.m}
% \lstinputlisting[language=Matlab]{capitolo5/simpcomp.m}
% Approssimando $\displaystyle \int_{-1}^{1.1}tan(x)dx$ con le due formule si ottiene:
% \begin{table}[h]
%     \begin{tabular}{|c|c|c |}
%     \hline
%     intervalli$\backslash$formula & trapezi composita         &simpson composita\\
%     \hline
%     $2$                           &$0.266403558406035$        &$0.266403558406035$ \\
%     $4$                           &$0.203432804450016$        &$0.182442553131343$ \\
%     $6$                           &$0.188498346613972$        &$0.177333443886033$ \\
%     $8$                           &$0.182789408875225$        &$0.175908277016961$ \\
%     $10$                          &$0.180034803521960$        &$0.175392868382289$ \\
%     $12$                          &$0.178504015707472$        &$0.175172572071972$ \\
%     $14$                          &$0.177568218195411$        &$0.175066546519247$ \\
%     $16$                          &$0.176955413111201$        &$0.175010747856527$ \\
%     $18$                          &$0.176532709616469$        &$0.174979254439942$ \\
%     $20$                          &$0.176229037552030$        &$0.174960448895386$ \\
%     \hline
% \end{tabular}
% \caption{risultati di \nameref{cod:24}}
% \label{tab:24}
% \end{table}


% La formula composita di simpson converge più rapidamente ed è più precisa rispetto alla formula dei trapezi