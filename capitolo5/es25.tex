\subsection{Esercizio 25}
Utilizzare le formule tabulate nel precedente esercizio per calcolare le approssimazioni dell'integrale
\[
    I(f) = \int_{0}^{1} e^{3x}\,dx,
\]
tabulando (in modo significativo) il corrispondente errore di quadratura (risolvere a mano l'integrale).
\newline \textbf{Soluzione:}

% \lstinputlisting{capitolo5/adaptrap.m}
% \lstinputlisting{capitolo5/adapsim.m}
% \newpage
% Approssimando $\displaystyle \int_{-1}^{1}\frac{1}{1+10^2x^2}dx$ con le due formule si ottiene:
% \begin{table}[h]
% \begin{tabular}{|c |c |c|}
%         \hline
%         tolleranza$\backslash$formula & trapezi adattiva          &simpson adattiva\\
%         \hline
%         $10^{-2}$                     &$0.295559711784128$,  punti $= 21$      &$0.281297643062670$,  punti $= 17$  \\
%         $10^{-3}$                     &$0.294585368185034$ , punti $= 93$     &$0.281297643062670$,  punti $= 17$  \\
%         $10^{-4}$                     &$0.294274200873635$,  punti $= 277$      &$0.294259338419631$, punti $= 41$  \\
%         $10^{-5}$                     &$0.294230142164878$, punti $= 793$      &$0.294227809768005$,  punti $= 81$  \\
%         $10^{-6}$                     &$0.294226019603178$,  punti $= 2692$      &$0.294225764620384$ ,  punti $= 145$ \\
%         \hline
%     \end{tabular}
%     \caption{risultati di \nameref{cod:25}}
%     \label{tab:25}
% \end{table}


% Per ciascuna formula, l'operazione che comporta maggior costo computazionale ad ogni chiamata è la valutazione funzionale dei punti di un sottointervallo.
% Poichè ogni punto viene valutato una sola volta, possiamo confrontare il costo delle due formule andando a vedere quanti punti aggiuntivi sono stati utilizzati.
% Osservando i dati riportati nella tabella \ref{tab:25}, è palese come la formula di simpson adattiva convergà più rapidamente rispetto alla formula dei trapezi adattiva.
