\subsection{Esercizio 21}
Costruire una tabella in cui viene riportato, al crescere di $n$, il massimo errore di
interpolazione ottenuto approssimando la funzione:
\[
    f(x) = \frac{1}{2(2x^2-2x+1)}
\]
sulle ascisse $x_0 \le x_1 \dots \le x_n$:
\begin{itemize}
    \item equidistanti in $[-2, 3]$,
    \item di Chebyshev per lo stesso intervallo,
\end{itemize}
utilizzando le function degli Esercizi 16-18 e 20, e la function $spline$ di Matlab. Considerare
$n = 4, 8, 16, \dots, 40$ e stimare l'errore di interpolazione su 10001 punti
equidistanti nell'intervallo $[x0, xn]$.
\newline \textbf{Soluzione:} \newline
% \lstinputlisting[language=Matlab]{./capitolo5/ncweights.m}
% Eseguendo lo script \nameref{cod:21} si ottiene:
% \begin{table}[h]
%     \centering
%     \renewcommand\arraystretch{3}
%     \begin{tabular}{|l|c|c|c|c|c|c|c|c|}
%         \hline
%         n $\backslash c_{in}$ & 0 & 1 & 2 & 3 & 4 & 5 & 6 & 7 \\
%         \hline
%         1 & $\dfrac{1}{2}$ & $\dfrac{1}{2}$ & & & & &  &\\
%         \hline
%         2 & $\dfrac{1}{3}$ & $\dfrac{4}{3}$ & $\dfrac{1}{3}$ & & & &&\\
%         \hline
%         3 &  $\dfrac{3}{8}$ & $\dfrac{9}{8}$ & $\dfrac{9}{8}$ & $\dfrac{3}{8}$ & & & & \\
%         \hline
%         4 & $\dfrac{14}{45} $ & $\dfrac{64}{45} $ & $\dfrac{8}{15} $ & $\dfrac{64}{45} $ &$\dfrac{14}{45} $ &&&\\
%         \hline
%         5 & $\dfrac{95}{288}$ & $\dfrac{125}{96}$ & $\dfrac{125}{144}$ & $\dfrac{125}{144}$ & $\dfrac{125}{96}$ & $\dfrac{95}{288}$ &&\\
%         \hline
%         6 & $\dfrac{41}{140}$ & $\dfrac{54}{35}$ & $\dfrac{27}{140}$ & $\dfrac{68}{35}$ & $\dfrac{27}{140}$ & $\dfrac{54}{35}$ & $\dfrac{41}{140}$ &\\
%         \hline
%         7 & $\dfrac{108}{355}$ & $\dfrac{810}{559}$ & $\dfrac{343}{640}$ & $\dfrac{649}{536}$ & $\dfrac{649}{536}$  & $\dfrac{343}{640}$ & $\dfrac{810}{559}$ & $\dfrac{108}{355}$\\
%         \hline
%     \end{tabular}
%     \caption{pesi della formula di Newton-Cotes fino al settimo grado}
% \end{table}