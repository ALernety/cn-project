% \newpage
\subsection{Esercizio 23}
Sia assegnata la seguente perturbazione della funzione $f(x) = sin(\Pi x^2)$:
\[
    \tilde{f}(x) = f(x) + 10^{-1} rand(size(x)),
\]
in cui $rand$ è la function built-in di Matlab. Calcolare polinomio di approssimazione ai minimi
quadrati di grado $m$, $p(x)$, sui dati $(x_i, \tilde{f}(x_i))$, $i = 0, \dots, n$, con:
\begin{eqnarray*}
    x_i = i/n, & & n = 10^4.
\end{eqnarray*}
Graficare (in formato $semilogy$) l'errore di approssimazione $\|f - p\|$ (stimato come il massimo
errore sui punti $x_i$), relativo all'intervallo $[0, 1]$, rispetto ad $m$, per $m = 1, 2, \dots, 15$.
Commentare i risultati ottenuti.
\newline \textbf{Soluzione:}

% \lstinputlisting{./capitolo5/newtoncotes.m}

% RISULTATI PER N DA 1 A 9(\nameref{cod:23}):

% \begin{table}[h]
%     \begin{tabular}{|l |l |l|}
%         \hline
%         grado della formula & valore integrale   & errore\\
%         \hline
%         1                   & $4,28\cdot10^{-1}$ & $2,53\cdot10^{-1}$\\
%         2                   & $2,13\cdot10^{-1}$ & $3,8\cdot10^{-2}$\\
%         3                   & $1,96\cdot10^{-1}$ & $2,1\cdot10^{-2}$\\
%         4                   & $1,80\cdot10^{-1}$ & $5\cdot10^{-3}$\\
%         5                   & $1,79\cdot10^{-1}$ & $4\cdot10^{-3}$\\
%         6                   & $1,76\cdot10^{-1}$ & $1\cdot10^{-3}$\\
%         7                   & $1,76\cdot10^{-1}$ & $1\cdot10^{-3}$\\
%         8                   & $1,75\cdot10^{-1}$ & $0$\\
%         9                   & $1,75\cdot10^{-1}$ & $0$\\
%         \hline
%     \end{tabular}
% \end{table}

