\subsection{Esercizio 26}
Scrivere una function Matlab,
\[
    [If, err, nfeval] = composita(fun, a, b, n, tol)
\]
in qui
\begin{itemize}
    \item $fun$ è l'identificatore di una function che calcoli (in modo vettoriale) la funzione integranda,
    \item $a$ e b sono gli estremi dell'intervallo di integrazione,
    \item $n$ è il grado di una formula di Newton-Cotes base,
    \item $tol$ è l'accuratezza richiesta,
\end{itemize}
che calcoli, fornendo la stima $err$ dell'errore di quadratura, l'approssimazione $If$ dell'integrale,
raddoppiando il numero di punti ed usando la formula composita corrispondente per stimare l'errore
di quadratura, fino a soddisfare il requisito di accuratezza richiesto. In uscita è anche il numero
totale di valutazioni funzionali effettuate, $nfeval$.
\newline N.B.: evitare di effettuare valutazioni di funzione ridondanti.
\newline \textbf{Soluzione:}

\lstinputlisting{matlab/capitolo5/composita.m}
