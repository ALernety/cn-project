\subsection{Esercizio 1} 
Sia $f(x)$ una funzione sufficientemente regolare e sia $h>0$ una quantità abbastanza ''piccola''. Possiamo sviluppare i termini $f(x-h)$ e $f(x+h)$ mediante il polinomio di Taylor:
\[
f(x+h) = f(x) +  hf'(x) + \frac{h^2}{2}f''(x) + \frac{h^3}{6}f'''(x) + O(h^4)
\]
\[
f(x-h) = f(x) -  hf'(x) + \frac{h^2}{2}f''(x) - \frac{h^3}{6}f'''(x) + O(h^4)
\]
Sostituiamo i termini  nell'espressione iniziale:
\[\frac{f(x-h) -2f(x) + f(x+h)}{h^2} = \]
\[
 =\frac{ f(x) -  hf'(x) + \frac{h^2}{2}f''(x) - \frac{h^3}{6}f'''(x) + O(h^4) -2f(x) + f(x) + hf'(x) + \frac{h^2}{2}f''(x) + \frac{h^3}{6}f'''(x) + O(h^4)}{h^2} = \]

\[=\frac{h^2f''(x) + O(h^4)}{h^2} = f''(x) + O(h^2)
\]


