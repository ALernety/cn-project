\subsection{Esercizio 1}
%%%%%
\newcommand\esOneFormula[4]
{\frac{-#1 + 8#2 - 8#3 + #4 + O(h^4)}{12h}}
%%%%%
Verificare che,
\[
    \esOneFormula{f(x+2h)}{f(x+h)}{f(x-h)}{f(x-2h)} = f'(x)+ O(h^4)
    % \frac{-f(x+2h) + 8f(x+h) - 8f(x-h) + f(x-2h) + O(h^4)}{12h} = f'(x)+ O(h^4)
\]
\newline \textbf{Soluzione:} \newline

Per verificare la equvalenza ci servirà applicare sviluppo di Taylor per tutti i funzioni.
La funzione di Taylor definita cosi:
\[
    f(x) = f(x_0) + f'(x_0)(x-x_0) + \frac{f''(x_0)}{2!}(x-x_0)^2 + ... + \frac{f^{(n)}(x_0)}{n!}(x-x_0)^n + O((x-x_0)^n)
\]
%%%%%
\newcommand\esOneTaylorTwoSteps[2]
{f(#2) + f'(#2)(#1-#2) + O(#1-#2)}
%%%%%
Calcoliamo le sviluppi di Taylor per tutti funzioni che sono presenti in formula da verificare.
A variablie $x_0$ noi assumiamo il valore $x$, quindi $x_0=x$. Allora:
\[
    f(x+2h) = \esOneTaylorTwoSteps{x+2h}{x} = f(x) + 2hf'(x) + O(h)
\]
\[
    f(x+h) = \esOneTaylorTwoSteps{x+h}{x} = f(x) + hf'(x) + O(h)
\]
\[
    f(x-h) = \esOneTaylorTwoSteps{x-h}{x} = f(x) - hf'(x) + O(h)
\]
\[
    f(x-2h) = \esOneTaylorTwoSteps{x-2h}{x} = f(x) - 2hf'(x) + O(h)
\]
Se mettere tutto insieme otteniamo:
% \begin{align*}
%     \esOneFormula{(f(x) + 2hf'(x))}{(f(x) + hf'(x))}{(f(x) - hf'(x))}{(f(x) - 2hf'(x))} = \\
%     = \frac{\cancel{-f(x)} - 2hf'(x) + \xcancel{8f(x)} + 8hf'(x)
%     - \xcancel{8f(x)} + 8hf'(x) + \cancel{f(x)} - 2hf'(x) + O(h^4)}{12h} =      \\
%     = \frac{\cancel{12h}f'(x) + O(h^4)}{\cancel{12h}} = f'(x) + O(h^4)
% \end{align*}
\[
    \esOneFormula{(f(x) + 2hf'(x))}{(f(x) + hf'(x))}{(f(x) - hf'(x))}{(f(x) - 2hf'(x))} =
\]
\[
    = \frac{
        \cancel{-f(x)} - 2hf'(x) 
        + \cancel{8f(x)} + 8hf'(x) 
        - \cancel{8f(x)} + 8hf'(x) 
        + \cancel{f(x)} - 2hf'(x) 
        + O(h^4)}
    {12h} =
\]
\[
    = \frac{\cancel{12h}f'(x) + O(h^4)}{\cancel{12h}} = f'(x) + O(h^4)
\]
