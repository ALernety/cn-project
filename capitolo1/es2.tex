\subsection{Esercizio 2}
Calcolare, motivandone i passaggi, la precisione di macchina della doppia precisione
dello standard IEEE. Confrontare questa quantità con quanto ritornato dalla variabile eps di
Matlab, commentando a riguardo.
\textbf{Soluzione:} \newline
Dato un generico numero reale $x \in I$, con $I$ sottoinsieme della retta reale, e
dato $\mathbf{M}$ come insieme dei numeri di macchina che appartengono a sottoinsieme
della retta reale $I$, sappiamo dalla teoria che il numero di elementi di $\mathbf{M}$ é finito.
Pertanto é necessario definire una funzione $fl: I V \mathbf{M}$, se $x \in I$ e $x \neq 0$ con
$fl(x) = x(1 + \varepsilon_x)$, dove $\varepsilon_x$ é l'errore relativo della funzione tale che $\varepsilon_x \leq u$,
\begin{equation*}
    u =\begin{cases}
        2^{1-m},             & per troncamento    \\
        \frac{1}{2}*2^{1-m}, & per arrotondamento
    \end{cases}
\end{equation*}
Poiché lo studio riguarda la doppia precisione, la mantissa $m$ è uguale a 53.
Per confrontare questi valori con l'$eps$ di Matlab, è necessario eseguire il seguente codice.
Le variabili $truncated$ e $rounded$ avranno rispettivamente i risultati del troncamento e dell'arrotondamento:
\begin{lstlisting}[language=Matlab]
>> a=eps
a = 2.2204e-16
>> truncated = 2^(1-53)
truncated = 2.2204e-16
>> rounded = (1/2)*(2^(1-53))
rounded = 1.1102e-16
\end{lstlisting}
Dai precedenti valori, è chiaro che l'$epsilon$ di macchina combacia con il valore
della precisione usato per il troncamento, mentre è il doppio esatto di quello per
l'arrotondamento.
