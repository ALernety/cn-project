\subsection{Esercizio 12}
Scrivere una function Matlab,
\begin{lstlisting}
function [x,nr] = miaqr(A, b)
\end{lstlisting}
che, data in ingresso la matrice $A$ $m \times n$, con $m \geq n = rank(A)$, ed un vettore $b$ di lunghezza
$m$, calcoli la soluzione del sistema lineare $Ax = b$ nel senso dei minimi quadrati e, inoltre, la
norma, $nr$, del corrispondente vettore residuo. Curare particolarmente la scrittura e l'efficienza della
\textit{function}. Validare la \textit{function} \lstinline{miaqr} su due esempi non banali,
generati casualmente, confrontando la soluzione ottenuta con quella calcolata con l'operatore Matlab.
\newline \textbf{Soluzione:}

\lstinputlisting{capitolo3/qr.m}
\begin{lstlisting}
>> m = randi(3)+2
m = 5
>> A = randi(10,m,m+1-randi(3))
A =

    1    1    4    3    2
    1   10    1    3    9
    8    2    5    6    1
    1    6    4    5   10
    8    7    9    3   10

>> b = randi(10,m,1)
b =

   10
    3
    8
    4
    5

>> [x, nr] = es12_qr(A, b)
x =

  -1.7272
   1.8092
   2.7103
   1.1353
  -2.1646

nr =

   5.3291e-15
   1.0658e-14
  -8.8818e-16
   7.1054e-15
            0

>> [c,r] = qr(A,b); r\c
ans =

  -1.7272
   1.8092
   2.7103
   1.1353
  -2.1646

>> ;
>> ;
>> ;
>> m = randi(3)+2
m = 4
>> A = randi(10,m,m+1-randi(3))
A =

   2   2
   9   4
   4   7
   9   7

>> b = randi(10,m,1)
b =

   10
    9
    6
    7

>> [x, nr] = es12_qr(A, b)
x =

   0.6783
   0.4927

nr =

  -7.6579
  -0.9242
   0.1622
   2.5539

>> [c,r] = qr(A,b); r\c
ans =

   0.6783
   0.4927

>>
\end{lstlisting}
Per dimostrare che risultato è corretto stato utilizzato il metodo discritto sul
\href{https://it.mathworks.com/help/matlab/ref/qr.html}{pagina di descrizione di funzione qr}
