\subsection{Esercizio 12}
Scrivere una function Matlab,
\begin{lstlisting}
function [x,nr] = miaqr(A, b)
\end{lstlisting}
che, data in ingresso la matrice $A$ $m \times n$, con $m \geq n = rank(A)$, ed un vettore $b$ di lunghezza
$m$, calcoli la soluzione del sistema lineare $Ax = b$ nel senso dei minimi quadrati e, inoltre, la
norma, $nr$, del corrispondente vettore residuo. Curare particolarmente la scrittura e l'efficienza della
\textit{function}. Validare la \textit{function} \lstinline{miaqr} su due esempi non banali,
generati casualmente, confrontando la soluzione ottenuta con quella calcolata con l'operatore Matlab.
\newline \textbf{Soluzione:}

\lstinputlisting{matlab/capitolo3/miaqr.m}
Eseguendo lo script \nameref{cod:12} si ottengono i risultati contenuti nelle
tabelle \ref{tab:12_1} e \ref{tab:12_2}.

Prima sistema lineare è
\[
   \begin{bmatrix}
      3  & 9 & 8 & 9 \\
      4  & 1 & 9 & 1 \\
      10 & 2 & 6 & 1 \\
      9  & 3 & 8 & 9
   \end{bmatrix}
   \begin{bmatrix}
      x_{1} \\
      x_{2} \\
      x_{3} \\
      x_{4}
   \end{bmatrix}
   =
   \begin{bmatrix}
      3 \\
      1 \\
      1 \\
      9
   \end{bmatrix}
\]
Invece seconda sistema lineare è
\[
   \begin{bmatrix}
      6  & 6 \\
      10 & 7 \\
      6  & 9
   \end{bmatrix}
   \begin{bmatrix}
      x_{1} \\
      x_{2}
   \end{bmatrix}
   =
   \begin{bmatrix}
      10 \\
      8  \\
      10
   \end{bmatrix}
\]
Eseguendo la funzione creata \lstinline{miaqr} e comando interno di
matlab \lstinline{r\c}, possiamo confrontare il risultati di funzione
scritta con quello realizzato nativamente in matlab.
\begin{table}[ht]
   \centering
   \renewcommand\arraystretch{2}
   \begin{tabular}{|l | c c |}
      \hline
      $x$     & $miaqr$                & $r \backslash c$       \\
      \hline
      $x_{1}$ & 1.491803278688525e-01  & 1.491803278688524e-01  \\
      $x_{2}$ & -8.508196721311473e-01 & -8.508196721311487e-01 \\
      $x_{3}$ & 1.475409836065585e-02  & 1.475409836065600e-02  \\
      $x_{4}$ & 1.121311475409836e+00  & 1.121311475409836e+00  \\
      \hline
   \end{tabular}
   \caption{Confronto di soluzioni di \lstinline{miaqr} e $r \backslash c$ per primo sistema}
   \label{tab:12_1}
\end{table}
\FloatBarrier
\begin{table}[ht]
   \centering
   \renewcommand\arraystretch{2}
   \begin{tabular}{| l | c c |}
      \hline
      $x$     & $miaqr$               & $r \backslash c$      \\
      \hline
      $x_{1}$ & 8.130081300812995e-02 & 8.130081300813012e-02 \\
      $x_{2}$ & 1.162601626016260e+00 & 1.162601626016260e+00 \\
      \hline
   \end{tabular}
   \caption{Confronto di soluzioni di \lstinline{miaqr} e $r \backslash c$ per secondo sistema}
   \label{tab:12_2}
\end{table}
Per dimostrare che risultato è corretto stato utilizzato il metodo discritto sul
\href{https://it.mathworks.com/help/matlab/ref/qr.html}{pagina di descrizione di funzione qr}
\FloatBarrier
