\subsection{Esercizio 13}
Utilizzare la \textit{function} $miaqr$ per risolvere, nel senso dei minimi quadrati,
i sistemi lineari sovradeterminati
\begin{eqnarray*}
   \mbox{A x = b,} & & \mbox{(D*A)x = (D*b)}
\end{eqnarray*}
definiti dai seguenti dati:
\begin{lstlisting}
A = [ 1 3 2; 3 5 4; 5 7 6; 3 6 4; 1 4 2 ];
b = [ 15 28 41 33 22 ]';
D = diag(1:5);
\end{lstlisting}
Commentare i risultati ottenuti.
\newline \textbf{Soluzione:}

Eseguendo lo script \nameref{cod:13} si ottengono i risultati contenuti nelle
tabelle \ref{tab:13_1} e \ref{tab:13_2}. Le norme dei vettori restituiti dei
sistemi \mbox{A x = b} e \mbox{(D*A)x = (D*b)} sono $2.999999999999994e+00$ e
$6.025699862322197e-01$ rispettivamente.
\begin{table}[ht]
   \centering
   \renewcommand\arraystretch{2}
   \begin{tabular}{|l | c c |}
      \hline
      $x$     & $miaqr$                & $A \backslash b$       \\
      \hline
      $x_{1}$ & -6.025699862322197e-01 & -6.025699862322226e-01 \\
      $x_{2}$ & 4.701698026617721e+00  & 4.701698026617714e+00  \\
      $x_{3}$ & 1.758375401560348e+00  & 1.758375401560356e+00  \\
      \hline
   \end{tabular}
   \caption{Confronto di soluzioni di \lstinline{miaqr} e $A \backslash b$ per \mbox{(D*A)x = (D*b)}}
   \label{tab:13_1}
\end{table}
\FloatBarrier
Si nota che \mbox{A x = b} e \mbox{(D*A)x = (D*b)} sono equivalenti.
Infatti partendo da \mbox{(D*A)x = (D*b)}:
\[
   \begin{bmatrix}
      1  & 3  & 2  \\
      6  & 10 & 8  \\
      15 & 21 & 18 \\
      12 & 24 & 16 \\
      5  & 20 & 10
   \end{bmatrix}
   =
   \begin{bmatrix}
      15  \\
      56  \\
      123 \\
      132 \\
      110
   \end{bmatrix}
\]
Raccogliendo fattori communi e portando in secondo membro, si ottiene:
\[
   \begin{bmatrix}
      1 & 3 & 2 \\
      3 & 5 & 4 \\
      5 & 7 & 6 \\
      3 & 6 & 4 \\
      1 & 4 & 2
   \end{bmatrix}
   =
   \begin{bmatrix}
      15 \\
      28 \\
      41 \\
      33 \\
      22
   \end{bmatrix}
\]
Che è esattamente \mbox{A x = b}.

Le soluzioni calcolati usando funzione scritta \lstinline{miaqr} e \lstinline{A\b}
non sono uguali, perché è una soluzione ai minimi quadrati.
\begin{table}[ht]
   \centering
   \renewcommand\arraystretch{2}
   \begin{tabular}{| l | c c |}
      \hline
      $x$     & $miaqr$                & $A \backslash b$       \\
      \hline
      $x_{1}$ & 2.999999999999994e+00  & 2.999999999999963e+00  \\
      $x_{2}$ & 5.799999999999997e+00  & 5.799999999999986e+00  \\
      $x_{3}$ & -2.499999999999987e+00 & -2.499999999999947e+00 \\
      \hline
   \end{tabular}
   \caption{Confronto di soluzioni di \lstinline{miaqr} e $A \backslash b$ per \mbox{A x = b}}
   \label{tab:13_2}
\end{table}
\FloatBarrier
