\subsection{Esercizio 13}
Utilizzare la \textit{function} $miaqr$ per risolvere, nel senso dei minimi quadrati,
i sistemi lineari sovradeterminati
\begin{eqnarray*}
    \mbox{A x = b,} & & \mbox{(D*A)x = (D*b)}
\end{eqnarray*}
definiti dai seguenti dati:
\begin{lstlisting}
A = [ 1 3 2; 3 5 4; 5 7 6; 3 6 4; 1 4 2 ];
b = [ 15 28 41 33 22 ]';
D = diag(1:5);
\end{lstlisting}
Commentare i risultati ottenuti.
\newline \textbf{Soluzione:}
\begin{lstlisting}
>> A = [ 1 3 2; 3 5 4; 5 7 6; 3 6 4; 1 4 2 ];
>> b = [ 15 28 41 33 22 ]';
>> D = diag(1:5);
>> [x,nr]=es12_qr(A,b)
x =

   3.0000
   5.8000
  -2.5000

nr =

   4.0000e-01
   2.1316e-14
  -4.0000e-01
   8.0000e-01
  -8.0000e-01

>> [c,r] = qr(A,b); r\c
ans =

   3.0000
   5.8000
  -2.5000

>> (D*A)*x
ans =

    15.400
    56.000
   121.800
   135.200
   106.000

>> D*b
ans =

    15
    56
   123
   132
   110

>>
\end{lstlisting}
Si vede che le soluzioni sono uguali per metodo implementato e quello di matlab,
però non rispettano la ugualianza richiesta in esercizio, anche se tendano per rispettare.
Questo dato da precisione di numeri con virgola mobile che sono usati in matlab.
