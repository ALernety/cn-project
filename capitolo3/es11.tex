\subsection{Esercizio 11}
Risolvere, utilizzando la \textit{function} \lstinline{mialdlt}, i sistemi
lineari generati da \lstinline{[A,b]=linsis(10, 1, 1)} e \lstinline{[A,b]=linsis(10, 10, 1)}.
Commentare l'accuratezza dei risultati ottenuti, dandone spiegazione esaustiva.
\newline \textbf{Soluzione:}

Eseguendo lo script \nameref{cod:11} si ottengono il risultato contenuto nella
tabella \ref{tab:11}.

\begin{table}[ht]
  \centering
  \renewcommand\arraystretch{2}
  \begin{tabular}{|l | c c |}
    \hline
    Risultato aspettato & $linsis(10, 1, 1)$    & $linsis(10, 10, 1)$ \\
    \hline
    1                   & 9.999999999999953e-01 &                     \\
    2                   & 2.000000000000004e+00 &                     \\
    3                   & 3.000000000000000e+00 & Errore!             \\
    4                   & 4.000000000000005e+00 & Matrice non è       \\
    5                   & 5.000000000000001e+00 & simmetrica definita \\
    6                   & 6.000000000000001e+00 & positiva            \\
    7                   & 6.999999999999997e+00 &                     \\
    8                   & 8.000000000000009e+00 &                     \\
    9                   & 9.000000000000005e+00 &                     \\
    10                  & 1.000000000000000e+01 &                     \\
    \hline
  \end{tabular}
  \caption{Confronto di soluzioni per \lstinline{linsis(10, 1)} e \lstinline{linsis(10, 10)} usando \lstinline{mialdl}}
  \label{tab:11}
\end{table}
\FloatBarrier
Nel \textbf{caso $linsis(10, 1, 1)$} notiamo che l'errore é molto piccolo.

Considerando che la soluzione con perturbazioni consiste nel risolvere il sistema
lineare $A(\epsilon)x(\epsilon) = b(\epsilon)$, dato che $\epsilon \approx 0 \rightarrow x(0) = x$
la soluzione senza perturbazione, o comunque dato $\epsilon \approx 0$ quasi senza perturbazione.
Infatti eseguendo il comando \lstinline{cond} otteniamo $9.999999999999996e+00$.

Nel \textbf{caso $linsis(10, 10, 1)$} ci da errore \lstinline{The matrix provided is not symmetric and postive definite!}
che significa che matrice non è simmetrica definita positiva.
Come nel esercizo precedente calcolando il numero di condizionamento
$K(A^T\cdot A) = \|A^T A\|\cdot\|(A^T A)^{-1}\|$ abbiamo che $K = 7.620046936967337e+32 $.
Avendo $K \gg 1$ si ha un mal condizionamento del problema.
