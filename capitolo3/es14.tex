\subsection{Esercizio 14}

\begin{table}[h]
\begin{tabular}{|l l|}
        \hline
        A${\}$b & (A’*A)${\}$(A’*b)\\
        \hline
        1.0000& 3.5759 \\
    	2.0000&-3.4624   \\
    	3.0000&  9.5151\\
    	4.0000&-1.2974\\
    	5.0000& 7.9574\\
    	6.0000& 4.9125\\
    	7.0000& 7.2378\\
    	8.0000& 7.9765\\
        \hline
\end{tabular}
\caption{valori approssimati}
\label{tab::1}     
\end{table}

L'espressione A\b risolve in matlab,  il sistema di equazioni lineari nella forma matriciale A*x=b per X. \\\
L'espressione (A'*A) \ (A'*b) impiega lo stesso operatore \, quindi risolve il sistema delle equazioni lineare delle due parentesi.
La Matrice A viene calcolata usando la funzione vander() che genera una matrice di tipo Vandermond, la quale è molto mal condizionata. Usando la funzione cond() sulla matrice A si ottiene una condizionamento pari a: 1.5428e+09. Eseguendo poi la moltiplicazione della prima parentesi tonda, il condizionamento è pari a: 4.4897e+18. Andando cosi ad eseguire una divisione tra una matrice mal condizionata e un vettore, il risultato presenta degli errori.