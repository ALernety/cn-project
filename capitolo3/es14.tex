\subsection{Esercizio 14}

\begin{table}[h]
\begin{tabular}{|l l|}
        \hline
        A$\backslash$b & (A’*A)$\backslash$(A’*b)\\
        \hline
        1.0000& 3.5759 \\
    	2.0000&-3.4624   \\
    	3.0000&  9.5151\\
    	4.0000&-1.2974\\
    	5.0000& 7.9574\\
    	6.0000& 4.9125\\
    	7.0000& 7.2378\\
    	8.0000& 7.9765\\
        \hline
\end{tabular}
\caption{valori approssimati}
\label{tab:14}     
\end{table}

L'espressione $A \backslash b$ risolve in matlab,  il sistema di equazioni lineari nella forma matriciale $A\cdot x=b$ per x.
L'espressione $(A^{T}\cdot A) \backslash (A^{T} \cdot b)$, è matematicamente la stessa operazione dell'espressione precedente, solamente che si moltiplica le due componenti per la trasposta di A. 
La matrice A viene calcolata usando la funzione vander() che genera una matrice di tipo Vandermonde, la quale è  mal condizionata. Usando la funzione cond() sulla matrice A si ottiene una condizionamento pari a: 1.5428e+09.Nella prima espressione, questo malcondizionamento non influisce sul risultato.Invece nella seconda espressione eseguendo la prima parentesi tonda, il condizionamento è pari a: 4.4897e+18. Questo fa si che eseguendo la divisione tra una matrice mal condizionata e il vettore, il risultato presenta degli errori.