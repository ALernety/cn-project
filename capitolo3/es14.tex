\subsection{Esercizio 14}
Scrivere una function Matlab,
\begin{lstlisting}
[x,nit] = newton(fun, jacobian, x0, tol, maxit)
\end{lstlisting}
che implementi efficientemente il metodo di Newton per risolvere sistemi di equazioni nonlineari.
Curare particolarmente il criterio di arresto, che deve essere analogo a quello usato nel caso scalare.
La seconda variabile, se specificata, ritorna il numero di iterazioni eseguite. Prevedere opportuni
valori di $default$ per gli ultimi due parametri di ingresso.
\newline \textbf{Soluzione:}
\lstinputlisting{matlab/capitolo3/newton.m}
