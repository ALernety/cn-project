\subsection{Esercizio 9}
Scrivere una function Matlab,
\begin{lstlisting}
function x = mialdl(A, b)
\end{lstlisting}
che, dati in ingresso una matrice sdp $A$ ed un vettore $b$, calcoli la soluzione
del corrispondente sistema lineare utilizzando la fattorizzazione $LDL^T$.
Curare particolarmente la scrittura e l'efficienza della function,
e validarla su due esempi non banali, generati casualmente, di cui sia nota la soluzione.
\newline \textbf{Soluzione:} \newline
\lstinputlisting{capitolo3/es9_ldl.m}
Per verificare che soluzioni sono esatte sto controllando risultati con comando \lstinline{A * solutions}
che moltiplica tutti colonne di \lstinline{A} con tutti rige di \lstinline{solutions} e alla fine somma tutti
elementi in riga che alla fine deve essere esattamente stesso numero
che contiene \lstinline{b} sul stessa riga.
\begin{lstlisting}
>> matrixDimension = 5;
>> A = round(10 * rand(matrixDimension));
>> A = A * A'
A =

   110    99   155    72   101
    99   158   188    46   109
   155   188   268    94   175
    72    46    94    56    78
   101   109   175    78   194

>> b = round(10 * rand(matrixDimension, 1))
b =

   8
   5
   2
   9
   5

>> solutions = es9_ldl(A, b)
solutions =

   -8.9866
    6.9667
   -3.6120
   14.5835
   -1.8151

>> A * solutions
ans =

   8
   5
   2
   9
   5

>> ;
>> ;
>> ;
>> matrixDimension = 2;
>> A = round(10 * rand(matrixDimension));
>> A = A * A'
A =

   68   58
   58   50

>> b = round(10 * rand(matrixDimension, 1))
b =

   2
   9

>> solutions = es9_ldl(A, b)
solutions =

  -11.722
   13.778

>> A * solutions
ans =

   2
   9

>>
\end{lstlisting}
