\subsection{Esercizio 9}
Scrivere una function Matlab,
\begin{lstlisting}
function x = mialdl(A, b)
\end{lstlisting}
che, dati in ingresso una matrice sdp $A$ ed un vettore $b$, calcoli la soluzione
del corrispondente sistema lineare utilizzando la fattorizzazione $LDL^T$.
Curare particolarmente la scrittura e l'efficienza della function,
e validarla su due esempi non banali, generati casualmente, di cui sia nota la soluzione.
\newline \textbf{Soluzione:}

\lstinputlisting{matlab/capitolo3/mialdl.m}
Eseguendo lo script \nameref{cod:9} si ottengono i risultati contenuti nelle
tabelle \ref{tab:9_1} e \ref{tab:9_2}.

Prima sistema lineare è
\[
   \begin{bmatrix}
      166 & 105 & 161 & 132 \\
      105 & 119 & 151 & 98  \\
      161 & 151 & 239 & 144 \\
      132 & 98  & 144 & 131
   \end{bmatrix}
   \begin{bmatrix}
      x_{1} \\
      x_{2} \\
      x_{3} \\
      x_{4}
   \end{bmatrix}
   =
   \begin{bmatrix}
      5  \\
      10 \\
      6  \\
      2
   \end{bmatrix}
\]
Invece seconda sistema lineare è
\[
   \begin{bmatrix}
      45 & 27 \\
      27 & 18
   \end{bmatrix}
   \begin{bmatrix}
      x_{1} \\
      x_{2}
   \end{bmatrix}
   =
   \begin{bmatrix}
      2 \\
      5
   \end{bmatrix}
\]
Eseguendo la funzione creata \lstinline{mialdl} e comando interno di
matlab \lstinline{A\b}, possiamo confrontare il risultati di funzione
scritta con quello realizzato nativamente in matlab.
\begin{table}[ht]
   \centering
   \renewcommand\arraystretch{2}
   \begin{tabular}{|l | c c |}
      \hline
      $x$     & $mialdl$               & $A \backslash b$       \\
      \hline
      $x_{1}$ & 9.752409097697867e-02  & 9.752409097697871e-02  \\
      $x_{2}$ & 2.974225536634357e-01  & 2.974225536634358e-01  \\
      $x_{3}$ & -1.315833858151011e-01 & -1.315833858151011e-01 \\
      $x_{4}$ & -1.608594100046056e-01 & -1.608594100046056e-01 \\
      \hline
   \end{tabular}
   \caption{Confronto di soluzioni di \lstinline{mialdl} e $A \backslash b$ per primo sistema}
   \label{tab:9_1}
\end{table}
\begin{table}[ht]
   \centering
   \renewcommand\arraystretch{2}
   \begin{tabular}{| l | c c |}
      \hline
      $x$     & $mialdl$               & $A \backslash b$       \\
      \hline
      $x_{1}$ & -1.222222222222222e+00 & -1.222222222222221e+00 \\
      $x_{2}$ & 2.111111111111110e+00  & 2.111111111111110e+00  \\
      \hline
   \end{tabular}
   \caption{Confronto di soluzioni di \lstinline{mialdl} e $A \backslash b$ per secondo sistema}
   \label{tab:9_2}
\end{table}
\FloatBarrier
