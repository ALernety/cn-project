\subsection{Esercizio 9}
Scrivere una function Matlab,
\begin{lstlisting}[language=Matlab]
function x = mialdl(A, b)
\end{lstlisting}
che, dati in ingresso una matrice sdp $A$ ed un vettore $b$, calcoli la soluzione
del corrispondente sistema lineare utilizzando la fattorizzazione $LDL^T$.
Curare particolarmente la scrittura e l'efficienza della function,
e validarla su due esempi non banali, generati casualmente, di cui sia nota la soluzione.
\newline \textbf{Soluzione:} \newline
\lstinputlisting[language=Matlab]{capitolo3/es9_ldl.m}
Per verificare che soluzioni sono giusti è stata creata la funzione \nameref{cod:testMatrixSolution}
che moltiplica valori di soluzione con la riga e somma tutti valori ottenuti
dopodiché stampa questo valore, che deve essere uguale allo valore di stessa riga di $b$.
\begin{lstlisting}[language=Matlab]
>> matrixDimension = 5;
>> A = round(10 * rand(matrixDimension));
>> A = A * A'
A =

   170   189   102   166    90
   189   278    96   203   115
   102    96    93   108    41
   166   203   108   195   103
    90   115    41   103    87

>> b = round(10 * rand(matrixDimension, 1))
b =

   2
   6
   3
   1
   4

>> solutions = es9_ldl(A, b)
solutions =

  -0.2009
   0.1788
   0.3425
  -0.3300
   0.2468

>> testMatrixSolution(A,solutions)
2.0000
6.0000
3.0000
1
4.0000
>> ;
>> ;
>> ;
>> matrixDimension = 2;
>> A = round(10 * rand(matrixDimension));
>> A = A * A'
A =

   162    81
    81    65

>> b = round(10 * rand(matrixDimension, 1))
b =

   2
   9

>> solutions = es9_ldl(A, b)
solutions =

  -0.1509
   0.3265

>> testMatrixSolution(A,solutions)
2
9
>>
\end{lstlisting}
