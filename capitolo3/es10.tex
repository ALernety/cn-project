\subsection{Esercizio 10}
\label{subsec:es10}
Data la function Matlab
\lstinputlisting{matlab/capitolo3/linsis.m}
che crea sistemi lineari casuali con soluzione nota, risolvere, utilizzando la \textit{function} \lstinline{mialu}, i sistemi
lineari generati da \lstinline{[A,b]=linsis(10,1)} e \lstinline{[A,b]=linsis(10,10)}. Commentare l'accuratezza dei
risultati ottenuti, dandone spiegazione esaustiva.
\newline \textbf{Soluzione:}

Eseguendo lo script \nameref{cod:10} si ottengono il risultato contenuto nella
tabella \ref{tab:10}.

\begin{table}[ht]
  \centering
  \renewcommand\arraystretch{2}
  \begin{tabular}{|l | c c |}
    \hline
    Risultato aspettato & $linsis(10, 1)$       & $linsis(10, 10)$       \\
    \hline
    1                   & 9.999999999999953e-01 & -3.076447708385681e+02 \\
    2                   & 2.000000000000004e+00 & 3.866526443048850e+02  \\
    3                   & 3.000000000000000e+00 & 1.843572773667685e+02  \\
    4                   & 4.000000000000005e+00 & -3.424372553841305e+02 \\
    5                   & 5.000000000000001e+00 & 1.855084828402995e+02  \\
    6                   & 6.000000000000001e+00 & 5.519206266267746e+02  \\
    7                   & 6.999999999999997e+00 & -1.377749853666923e+02 \\
    8                   & 8.000000000000009e+00 & 7.063244889082213e+01  \\
    9                   & 9.000000000000005e+00 & -4.210526315789573e+00 \\
    10                  & 1.000000000000000e+01 & -6.080000000000002e+02 \\
    \hline
  \end{tabular}
  \caption{Confronto di soluzioni per \lstinline{linsis(10, 1)} e \lstinline{linsis(10, 10)} usando \lstinline{mialu}}
  \label{tab:10}
\end{table}
\FloatBarrier
Nel \textbf{caso $linsis(10, 1)$} notiamo che l'errore é molto piccolo.

Considerando che la soluzione con perturbazioni consiste nel risolvere il sistema
lineare $A(\epsilon)x(\epsilon) = b(\epsilon)$, dato che $\epsilon \approx 0 \rightarrow x(0) = x$
la soluzione senza perturbazione, o comunque dato $\epsilon \approx 0$ quasi senza perturbazione.
Infatti eseguendo il comando \lstinline{cond} otteniamo $1.000000000000000e+01$.

Nel \textbf{caso $linsis(10, 10)$} calcolando il numero di condizionamento
$K(A^T\cdot A) = \|A^T A\|\cdot\|(A^T A)^{-1}\|$ abbiamo che $K = 6.345578386010926e+19 $.
Avendo $K \gg 1$ si ha un mal condizionamento del problema.
