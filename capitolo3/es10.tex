\subsection{Esercizio 10}

\begin{table}[h]
\begin{tabular}{|l l l|}
        \hline
        i & Sigma  &  Norma\\
        \hline
        1 & $10^{-1}$  &  8.9839e-15\\
        2 & $10^{1}$ &  1.4865e-14 \\
        3 & $10^{3}$ &  1.3712e-12 \\
        4 & $10^{5}$ &  1.2948e-10 \\
        5 & $10^{7}$ &  5.3084e-09 \\
        6 & $10^{9}$ &  1.0058e-06 \\
        7 & $10^{11}$ &  8.5643e-05 \\
        8 & $10^{13}$ &  0.0107 \\
        9 & $10^{15}$ &  0.9814 \\
        10 & $10^{17}$ & 4.1004e+03\\
        \hline
\end{tabular}
Tabella che composta:
	i=indice dell'iterazione;
	Sigma=valore calcolato e usato dalla funzione linsis(), per introdurre un errore nella matrice generata A e nel suo vettore dei termini noti b, che cresce al crescere dell'interazione.
	Norma= valore della distanza tra il vettore x, soluzione del sistema lineare LU*x=b, il quale è affetto da errore, e il vettore xref, soluzione corretta del sistema.
Da questa tabella quindi si può notare come all'incremento della interazione, e quindi della sigma, l'errore nelle soluzioni  cresce, quasi proporzionalmente come sigma, con un fattore di $10^{2}$;
\caption{valori approssimati}
\label{tab::1}     
\end{table}
