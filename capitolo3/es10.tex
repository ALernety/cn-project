\subsection{Esercizio 10}
\label{subsec:es10}
Data la function Matlab
\lstinputlisting{matlab/capitolo3/linsis.m}
che crea sistemi lineari casuali con soluzione nota, risolvere, utilizzando la \textit{function} \lstinline{mialu}, i sistemi
lineari generati da \lstinline{[A,b]=linsis(10,1)} e \lstinline{[A,b]=linsis(10,10)}. Commentare l'accuratezza dei
risultati ottenuti, dandone spiegazione esaustiva.
\newline \textbf{Soluzione:}

\begin{lstlisting}
>> [A, b] = linsis(10, 1); es8_lu(A,b)
ans =

    1
    2
    3
    4
    5
    6
    7
    8
    9
   10

>> [A, b] = linsis(10, 10); es8_lu(A,b)
ans =

  -307.6448
   386.6526
   184.3573
  -342.4373
   185.5085
   551.9206
  -137.7750
    70.6324
    -4.2105
  -608.0000

>>
\end{lstlisting}
Nel \textbf{primo caso} notiamo che l'errore é uguale a zero, infatti abbiamo che i soluzioni di $A$
sono tutti come aspettati: \lstinline{ans = [1; 2; 3; 4; 5; 6; 7; 8; 9; 10]}
\newline
Considerando che la soluzione con perturbazioni consiste nel risolvere il sistema
lineare $A(\epsilon)x(\epsilon) = b(\epsilon)$, dato che $\epsilon = 0 \rightarrow x(0) = x$
la soluzione senza perturbazione.
\newline
\newline
Nel \textbf{secondo caso} calcolando il numero di condizionamento
$K(A^T\cdot A) = \|A^T A\|\cdot\|(A^T A)^{-1}\|$ abbiamo che $K = 10^{16}$
Avendo $k \gg 1$ si ha un mal condizionamento del problema.
