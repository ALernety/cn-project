\subsection{Esercizio 10}
Si nota dalla seguente tabella che sigma e la norma euclidea tra x(vettore delle incognite calcolate con la funzione LUsolve) e xref(valori esatti), sono direttamente proporzionali.
\begin{table}[h]
\begin{tabular}{|l l l l l l l l l l l l|}
        \hline
        Sigma &$10^{-1}$ & $10^{1}$&$10^{3}$&$10^{5}$&$10^{7}$&$10^{9}$&$10^{11}$&$10^{13}$&$10^{15}$&$10^{17}$\\
        \hline
        Norma &8.9839e-15&1.4865e-14&1.3712e-12&1.2948e-10&5.3084e-09&1.0058e-06&8.5643e-05&0.0107&0.9814&   4.1004e+03\\
        
        \hline
\end{tabular}
\caption{valori approssimati}
\label{tab::1}     
\end{table}
