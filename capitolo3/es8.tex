\subsection{Esercizio 8}
Scrivere una function Matlab,
\begin{lstlisting}
function x = mialu(A, b)
\end{lstlisting}
che, data in ingresso una matrice $A$ ed un vettore $b$, calcoli la soluzione
del sistema lineare $Ax = b$ con il metodo di fattorizzazione $LU$ con \textit{pivoting} parziale.
Curare particolarmente la scrittura e l'efficienza della function,
e validarla su due esempi non banali, generati casualmente,
di cui sia nota la soluzione.
\newline \textbf{Soluzione:}

\lstinputlisting{capitolo3/lu.m}
Per verificare che soluzioni sono esatte sto controllando risultati con comando \lstinline{A * solutions}
che moltiplica tutti colonne di \lstinline{A} con tutti rige di \lstinline{solutions} e alla fine somma tutti
elementi in riga che alla fine deve essere esattamente stesso numero
che contiene \lstinline{b} sul stessa riga.
\begin{lstlisting}
>> matrixDimension = randi(5);
>> A = randi(10, matrixDimension)
A =

   2   6   5   3
   7   5   9   7
   3   9   0   1
   5   4   1   3

>> b = randi(10, matrixDimension, 1)
b =

   3
   7
   9
   10

>> solutions = es8_lu(A, b)
solutions =

   0.2214
   0.6183
   -1.8931
   2.7710

>> A * solutions
ans =

   3
   7
   9
   10

>> ;
>> ;
>> ;
>> matrixDimension = randi(5);
>> A = randi(10, matrixDimension)
A =

   1   7   2
   6   7   7
   1   3   2

>> b = randi(10, matrixDimension, 1)
b =

   4
   6
   9

>> solutions = es8_lu(A, b)
solutions =

   -11.9500
   -1.2500
   12.3500

>> A * solutions
ans =

   4
   6
   9

>>
\end{lstlisting}
