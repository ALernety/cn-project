\subsection{Esercizio 8}
Scrivere una function Matlab,
\begin{lstlisting}[language=Matlab]
function x = mialu(A, b)
\end{lstlisting}
che, data in ingresso una matrice $A$ ed un vettore $b$, calcoli la soluzione
del sistema lineare $Ax = b$ con il metodo di fattorizzazione $LU$ con \textit{pivoting} parziale.
Curare particolarmente la scrittura e l'efficienza della function,
e validarla su due esempi non banali, generati casualmente,
di cui sia nota la soluzione.
\newline \textbf{Soluzione:} \newline
\lstinputlisting[language=Matlab]{capitolo3/es8_lu.m}
Per verificare che soluzioni sono giusti è stata creata la funzione \nameref{cod:testMatrixSolution}
che moltiplica valori di soluzione con la riga e somma tutti valori ottenuti
dopodiché stampa questo valore, che deve essere uguale allo valore di stessa riga di $b$.
\begin{lstlisting}[language=Matlab]
>> matrixDimension = 4;
>> A = round(10 * rand(matrixDimension))
A =

    1    4    6    4
    3   10    2    8
    8    7    7    0
   10    4    6    1

>> b = round(10 * rand(matrixDimension, 1))
b =

    8
    3
   10
    6

>> solutions = es8_lu(A, b)
solutions =

  -0.3681
   0.4884
   1.3609
  -0.4377

>> testMatrixSolution(A,solutions)
8
3.0000
10.0000
6.0000
>> ;
>> ;
>> ;
>> matrixDimension = 3;
>> A = round(10 * rand(matrixDimension))
A =

   9   5   5
   9   0   3
   2   1   2

>> b = round(10 * rand(matrixDimension, 1))
b =

   1
   8
   7

>> solutions = es8_lu(A, b)
solutions =

  -1.4762
  -4.2381
   7.0952

>> testMatrixSolution(A,solutions)
1
8
7
>>
\end{lstlisting}
